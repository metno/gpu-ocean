\documentclass[11pt, a4paper]{article} 
\usepackage[norsk, english]{babel}
\usepackage[latin1]{inputenc}
\usepackage{graphicx}      
\usepackage{amsmath}
%\setcounter{secnumdepth}{0}
\usepackage{tikz}
\usetikzlibrary{arrows, automata}
\usepackage{color}
\usepackage{hyperref}
\usepackage{subcaption}
\usepackage{standalone}
\usepackage{algorithm2e}
\usepackage{bm}



\usepackage[numbers]{natbib}
\usepackage[marginpar]{todo}
    

\setlength{\textheight}{240mm} 
\setlength{\textwidth}{150mm}   
\topmargin -5mm 
\oddsidemargin 5mm

\bibliographystyle{unsrt}



%% My own commands
\newcommand{\drm}[0]{\,\mathrm{d}}
\newcommand{\vecQ}[0]{\bm{Q}}
\newcommand{\vecU}[0]{\bm{U}}
\newcommand{\vecV}[0]{\bm{V}}
\newcommand{\vecn}[0]{\bm{n}}
\newcommand{\tilvecQ}[0]{\tilde{\vecQ}}
\newcommand{\vecF}[0]{\bm{F}}
\newcommand{\vecS}[0]{\bm{S}}
\newcommand{\vecG}[0]{\bm{G}}
\newcommand{\vecq}[0]{\bm{q}}
\newcommand{\vecp}[0]{\bm{p}}
\newcommand{\dx}[0]{\Delta x}
\newcommand{\dy}[0]{\Delta y}

\newcommand{\partder}[2]{\frac{\partial #1}{\partial #2}}
\newcommand{\red}[1]{{\color{red}#1}}
\newcommand{\green}[1]{{\color{deepgreen}#1}}
\newcommand{\blue}[1]{{\color{blue}#1}}

\newcommand{\eastface}[1]{#1_{j + 1/2, k}}
\newcommand{\westface}[1]{#1_{j - 1/2, k}}
\newcommand{\northface}[1]{#1_{j, k + 1/2}}
\newcommand{\southface}[1]{#1_{j, k - 1/2}}

\newcommand{\eref}[1]{Equation (\ref{eq:#1})}


%%%%%%%%%%%%%%%%%%%%%%%%%%%%%%%%%%%%%%%%%%%%%%%%%%%%%%%%%%%%%%%%%%%%%%%%%
\begin{document}

\title{Reformulation of the CDKLM16 scheme} 
\author{H{\aa}vard Heitlo Holm}
\date{\today}

\maketitle

%\tableofcontents

\section{Introduction and Background}

We here consider the shallow water equations, which describe the behaviour of free surface waves over a varying bottom topography, where motion is due to gravity and Coriolis forces.
They are derived from the Navier-Stokes equations by the assumptions that the vertical water velocities are negligible compared to the velocities in the horizontal plane.
The equations are given as 
\begin{equation}
	\left[ \begin{matrix} h \\ hu \\ hv \\ \end{matrix} \right]_t
	 + \left[ \begin{matrix} hu \\ hu^2 + \frac{1}{2} gh^2 \\ huv \end{matrix} \right]_x
	 + \left[ \begin{matrix} hv \\ huv \\ hv^2 + \frac{1}{2} gh^2 \end{matrix} \right]_y 
	 = \left[ \begin{matrix} 0 \\ fvu \\ -fhu \end{matrix} \right]
	 + \left[ \begin{matrix} 0 \\ -ghB_x \\ -ghB_y \end{matrix} \right],
	 \label{eq:swe}
\end{equation}
where $h$ is the water depth, $u$ and $v$ are water velocities in $x$ and $y$ directions respectively, $g$ is the gravitational constant, and $f$ is the Coriolis parameter.
The source terms on the right hand side of the equation, accounts for the Coriolis forces and bottom topography $B = B(x,y)$, respectively.
The subscripts $t$, $x$ and $y$ denote derivatives with the respective variables.
Further, the Coriolis parameter is defined as a linear function of $y$, given by
\begin{equation}
	f(y) = \tilde{f} + \beta y.
	\label{eq:coriolis}
\end{equation}
The system in \eref{swe} can be written in the recognizable form of a hyperbolic conservation law in vector form, by defining $\vecq = [h, hu, hv]^T$ to be the vector of conserved variables, as
\begin{equation}
	\vecq_t  + \vecF(\vecq)_x + \vecG(\vecq)_y = \vecS_C(\vecq) +  \vecS_B(\vecq).
	\label{eq:swevec}
\end{equation}


\red{Add a paragraph on what qualities describes a good numerical scheme. E.g., capture steady state, deal with dry states (?), etc.}

Most existing numerical schemes for the shallow water equations are developed without the Coriolis source term, and aim at being \emph{well-balanced} in order to capture the lake-at-rest steady state solution
\begin{equation}
	u \equiv 0, \qquad v \equiv 0, \qquad h+B \equiv 0.
	\label{eq:lakeAtRest}
\end{equation}
This steady-state is preserved by careful reconstructions of the conserved variable $\vecq$ (see e.g., \cite{kurganovPetrova2007} \red{and more}).
In the presence of Coriolis force, more complex steady state solutions are present, and other reconstructions are required to preserve them.
Chertock et al. \cite{CDKLM16} presented a well-balanced scheme (hereby referred to as CDKLM) for capturing geostrophic equilibrium states, defined by
\begin{equation}
	u_x + v_y = 0, \qquad g(h+B)_x = fv, \qquad g(h+B)_y = -fu.
	\label{eq:geoEq}
\end{equation}
By introducing the primitives of the Coriolis force $[U, V]^T$, defined by
\begin{equation}
	V_x := \frac{f}{g}v, \qquad U_y := \frac{f}{g}u,
	\label{eq:coriolisPrimitives}
\end{equation}
and the potential energies
\begin{equation}
	K := g(h + B - V), \qquad L := g(h + B + U),
	\label{eq:coriolisPotEnergies}
\end{equation}
\eref{geoEq} can be rewritten to 
\begin{equation}
	u_x + v_y  = 0, \qquad K_x = 0, \qquad L_y = 0.
	\label{eq:geoEq2}
\end{equation}
The well-balanced piecewise linear reconstruction in CDKLM is therefore based on $\vecp = [u, v, K, L]^T$ instead of the conservative $\vecq$.

For applications of the shallow water equations in the presence of uncertainty in parameters and initial/boundary conditions, efficient simulation is a necessity.
Several efficient simulators for the shallow water equations have earlier been implemented on GPUs (see \cite{brodtkorb_etal_10_cvs}/ \cite{brodtkorb_etal_11_caf}, \cite{horvath_kepplerShuffle_2016}, \red{etc.}), and proven to give good performance gain in comparison to multi-threaded CPU implementations.
The CDKLM scheme, however, is formulated with a recursive term for $U$ and $V$ evaluated on the cell faces, and such calculations maps badly to the GPU architecture. 
In the following, we present a new formulation for the CDKLM16 scheme, which avoids the recursive terms and allows for efficient implementations on GPUs.


\section{Recursive formulation of the CDKLM scheme}
We consider a regular Cartesian discretization of a rectangular domain, with step sizes $\Delta x$ and $\Delta y$. 
Cell $C_{j,k}$ is centred at $(x_j, y_k)$, and is defined as $C_{j,k} := [x_{j-1/2}, x_{j+1/2}] \times [y_{k-1/2}, y_{k+1/2}]$, for $j = 0, ..., N_x$ and $k = 0, ..., N_y$.

The bottom topography are stored at on the cell intersections, $B_{j-1/2, k-1/2} = B(x_{j-1/2}, y_{k-1/2})$ for $j = 0, ..., N_x +1$ and $k = 0,..., N_y +1$.
It is then reconstructed as a continuous piecewise bilinear function, as described in \cite{kurganovPetrova2007}, so that the values on the cell faces becomes
\begin{equation}
	\eastface{B} = \frac{1}{2}(B_{j+1/2, k+1/2} + B_{j+1/2, k-1/2}),
	\label{eq:Bxfaces}
\end{equation}
and\begin{equation}
	\northface{B} = \frac{1}{2}(B_{j+1/2, k+1/2} + B_{j-1/2, k+1/2}).
	\label{eq:Byfaces}
\end{equation}
In the cell centres, the bottom becomes 
\begin{equation}
	B_{j,k} = \frac{1}{4} \left( \eastface{B} + \westface{B}+ \northface{B} + \southface{B} \right)
	\label{eq:Bm}
\end{equation}

The discretized conserved variables are defined as cell averages, as
\begin{equation}
	\vecq_{j,k} = \frac{1}{\dx \dy} \iint_{C_ {j,k}} \vecq(x, y) \drm x \drm y.
	\label{eq:cellAverage}
\end{equation}
The reconstruction variables for the velocities are then given as 
\begin{equation}
	u_{j,k} = \frac{(hu)_{j,k}}{h_{j,k}}, \quad \mathrm{and} \quad v_{j,k} = \frac{(hv)_{j,k}}{h_{j,k}}.
	\label{eq:uandv}
\end{equation}
\todo[small $h$?]{Make comment on small $h$?}
In order to get point values of $K$ and $L$, \cite{CDKLM16} defines $U$ and $V$ recursively on the cell faces, as
\begin{equation}
	U_{j, 0-1/2} = 0, \qquad 
	\northface{U} = \southface{U} + \frac{f_k}{g}u_{j,k} \dy,
	\label{eq:recursiveU}
\end{equation}
\begin{equation}
	V_{0-1/2, k} = 0, \qquad 
	\eastface{V} = \westface{V} + \frac{f_k}{g}v_{j,k} \dx,
	\label{eq:recursiveV}
\end{equation}
where $f_k = \tilde{f} + \beta y_k$.
The values for $U$ and $V$ in the cell centres is then 
\begin{equation}
	U_{j,k} = \frac{1}{2} \left(\eastface{U} + \westface{U} \right)
	\label{eq:Um}
\end{equation}
and
\begin{equation}
	V_{j,k} = \frac{1}{2} \left(\northface{V} + \southface{V} \right).
	\label{eq:Vm}
\end{equation}
The cell centre values for $K$ and $L$ can then be evaluated as a cell centred version of \eref{coriolisPotEnergies},
\begin{equation}
	K_{j,k} = g \left(h_{j,k} + B_{j,k} - V_{j,k} \right), \quad \mathrm{and} \quad 
	L_{j,k} = g \left(h_{j,k} + B_{j,k} + U_{j,k} \right).
	\label{eq:KandL}
\end{equation}

Based on the equilibrium variables in $\vecp_{j,k}$ from \eref{uandv} and \eref{KandL}, a piecewise linear approximation is constructed as 
\begin{equation}
	\tilde{\vecp}(x,y) = \vecp_{j,k} + (\vecp_x)_{j,k}(x - x_j) + (\vecp_y)_{j,k}(y - y_k), \qquad (x, y) \in C_{j,k}
	\label{eq:equilibriuimReconstruction}
\end{equation}
which is used to find reconstructed point values on the cell faces.
These are given by
\begin{equation}
	\begin{split}
		\vecp_{j,k}^E = \vecp_{j,k} + \frac{\dx}{2} \left( \vecp_{x} \right)_{j,k}, &\quad
		\vecp_{j,k}^W = \vecp_{j,k} - \frac{\dx}{2} \left( \vecp_{x} \right)_{j,k} \\
		\vecp_{j,k}^N = \vecp_{j,k} + \frac{\dy}{2} \left( \vecp_{y} \right)_{j,k},&\quad
		\vecp_{j,k}^S = \vecp_{j,k} - \frac{\dy}{2} \left( \vecp_{y} \right)_{j,k},
	\end{split}
	\label{eq:pOnFaces}
\end{equation}
where the derivatives values are found through the generalized minmod operator, which in the $x$-direction is
\begin{equation} 
	\begin{split}
	\left( \vecp_x \right)_{j,k} &= \mathrm{minmod}\left( \theta \frac{\vecp_{j+1, k} - \vecp_{j,k}}{\dx}, 
										   		\frac{\vecp_{j+1, k} - \vecp_{j-1,k}}{2\dx},
										    	\theta \frac{\vecp_{j, k} - \vecp_{j-1,k}}{\dx}  \right), \\
	\left( \vecp_y \right)_{j,k} &= \mathrm{minmod}\left( \theta \frac{\vecp_{j, k+1} - \vecp_{j,k}}{\dy}, 
										   		\frac{\vecp_{j, k+1} - \vecp_{j,k-1}}{2\dy},
										    	\theta \frac{\vecp_{j, k} - \vecp_{j,k-1}}{\dy}  \right).
	\end{split}
	\label{eq:pDerivatives}
\end{equation}
The minmod function is given by
\begin{equation*}
	\mathrm{minmod}(z_1, z_2, ...) := \begin{cases}
		\min(z_1, z_2, ...), \quad 	&\mathrm{if} \, z_i > 0 \; \forall \, i, \\
		\max(z_1, z_2, ...), 		&\mathrm{if}\, z_i < 0 \; \forall \, i \\
		0,\quad 				&\mathrm{otherwise}
	\end{cases}
\end{equation*}

The face values found in \eref{pOnFaces} will be used to numerically evaluate the flux contribution form both cells on each face.
The reconstructed values of $u$ and $v$ can be used directly, while the reconstructions for $K$ and $L$ are further used to reconstruct the water elevation, $h$, as
\begin{equation}
	\begin{split}
		h_{j,k}^E = \frac{K_{j,k}^E}{g} + \eastface{V} - \eastface{B} , &\quad
		h_{j,k}^W = \frac{K_{j,k}^W}{g} + \westface{V} - \westface{B} \\
		h_{j,k}^N = \frac{L_{j,k}^N}{g} - \northface{U} - \northface{B}, &\quad
		h_{j,k}^S = \frac{L_{j,k}^S}{g} - \southface{U} - \southface{B}.
	\end{split}
	\label{eq:hOnFaces}
\end{equation}



\subsection{Non-recursive formulation of the CDKLM scheme}
In order to implement the CDKLM scheme efficiently on modern massively parallel hardware, such as graphic processing units (GPUs), the aim is to formulate the above method without the recursive formula given in \eref{recursiveU} and \eref{recursiveV}.
First, note that the reconstructed water elevations in \eref{hOnFaces} rely on $L$ reconstructed in $y$-direction only, and $K$ reconstructed in $x$-direction only.
Computations of $K_y$ and $L_x$ are therefore superfluous.
Secondly, note that $K$ and $L$ represent potential forces, and that the forward, central and backward differences used in the minmod operator in \eref{pDerivatives} use the difference in these potentials, cancellation of recursive terms might allow us to formulate non-recursive versions of the scheme.
Third, if we are successful with non-recursive formulations of the $\left(\vecp_x\right)_{j,k}$ and $\left(\vecp_y\right)_{j,k}$, the reconstructed point values on the cell faces given in \eref{pOnFaces}, the reconstructed water elevations in \eref{hOnFaces} might be formulated similarly.

%Secondly, by inserting the recursive terms for $U$ and $V$ given in Equations \eqref{eq:recursiveU} and \eqref{eq:recursiveV}, 

\subsubsection{Evaluation of $(K_x)_{j,k}$}

By inserting $K_{j,k}$ into the \eref{pDerivatives}, $(K_x)_{j,k}$ is given by
\begin{equation}
	\left( K_x \right)_{j,k} = \mathrm{minmod}\left( \theta \frac{K_{j+1, k} - K_{j,k}}{\dx}, 
										   		\frac{K_{j+1, k} - K_{j-1,k}}{2\dx},
										    	\theta \frac{K_{j, k} - K_{j-1,k}}{\dx}  \right).
	\label{eq:KDerivative}
\end{equation}
By \eref{KandL}, the backward difference is written out as 
\begin{equation}
	\theta \frac{K_{j,k} - K_{j-1,k}}{\dx}  = \frac{\theta g}{\dx} \left( h_{j,k} + B_{j,k} - V_{j,k} - h_{j-1,k} - B_{j-1,k} + V_{j-1,k} \right),
	\label{eq:Kbackward1}
\end{equation}
where
\begin{equation}
	\begin{split}
		- V_{j,k} + V_{j-1,k} 
		\overset{\eqref{eq:Vm}}{=}& \frac{1}{2}\left(-V_{j-1/2,k} - V_{j+1/2, k} + V_{j- 1 -1/2, k} + V_{j-1/2, k}  \right)\\
		=\,&  \frac{1}{2}\left(- V_{j+1/2, k} + V_{j- 1 -1/2, k} \right) \\
		\overset{\eqref{eq:recursiveV}}{=}& \frac{1}{2} \left( - V_{j-1/2, k} - \frac{f}{g}v_{j,k}\dx + V_{j- 1 -1/2, k} \right) \\
		\overset{\eqref{eq:recursiveV}}{=}& \frac{1}{2} \left( - V_{j-1-1/2, k} - \frac{f}{g}v_{j-1, k}\dx - \frac{f}{g}v_{j,k}\dx + V_{j- 1 -1/2, k} \right) \\
		=\,&  - \frac{f}{2g} \left(v_{j-1,k} + v_{j,k} \right) \dx.
	\end{split}
	\label{eq:Kbackward2}
\end{equation}
Inserting \eref{Kbackward2} back into \eref{Kbackward1} results in 
\begin{equation}
	\theta \frac{K_{j,k} - K_{j-1,k}}{\dx} = \frac{\theta g}{\dx}\left(h_{j,k} + B_{j,k} - h_{j-1,k} - B_{j-1,k} - \frac{\dx f}{2g} \left( v_{j,k} + v_{j-1,k} \right) \right).
	\label{eq:Kbackwards}
\end{equation}
By the same arguments, the forward difference becomes
\begin{equation}
	\theta \frac{K_{j+1,k} - K_{j,k}}{\dx} = \frac{\theta g}{\dx}\left(h_{j+1,k} + B_{j+1,k} - h_{j,k} - B_{j,k} - \frac{\dx f}{2g} \left( v_{j+1,k} + v_{j,k} \right) \right).
	\label{eq:Kforwards}
\end{equation}

Similarly, the central difference can be written out
\begin{equation}
	\frac{K_{j+1,k} - K_{j-1,k}}{2\dx}  = \frac{g}{2\dx} \left( h_{j+1,k} + B_{j+1,k} - V_{j+1,k} - h_{j-1,k} - B_{j-1,k} + V_{j-1,k} \right),
	\label{eq:Kcentral1}
\end{equation}
where
\begin{equation}
	\begin{split}
		- V_{j+1,k} + V_{j-1,k} 
		\overset{\eqref{eq:Vm}}{=}& \frac{1}{2}\left(-V_{j+1+1/2,k} - V_{j+1/2, k} + V_{j- 1 -1/2, k} + V_{j-1/2, k}  \right)\\
		\overset{\eqref{eq:recursiveV}}{=}& \frac{1}{2} \left( - V_{j+1/2,k} - \frac{f}{g}v_{j+1,k}\dx - V_{j-1/2, k} - \frac{f}{g}v_{j,k}\dx + V_{j-1/2, k} + V_{j- 1 -1/2, k} \right) \\
		\overset{\eqref{eq:recursiveV}}{=}& \frac{1}{2} \left( - V_{j-1/2,k} - \frac{f}{g}v_{j,k}\dx - \frac{f}{g}v_{j+1,k}\dx - V_{j-1-1/2, k} - \frac{f}{g}v_{j-1,k}\dx - \frac{f}{g}v_{j,k}\dx \right.  \\
			& \;\quad\left. + V_{j-1/2, k} + V_{j- 1 -1/2, k} \vphantom{\frac{f}{g}} \right) \\
		=\,&  - \frac{f}{2g} \left(v_{j+1,k} + 2 v_{j,k} + v_{j-1,k} \right) \dx.
	\end{split}
	\label{eq:Kcentral2}
\end{equation}
Here, \eref{recursiveV} is applied twice to the two $V$ terms with the largest indices.
Inserting \eref{Kcentral2} back into \eref{Kcentral1} results in 
\begin{equation}
	\frac{K_{j+1,k} - K_{j-1,k}}{2\dx} = \frac{g}{2\dx} \left( h_{j+1,k} + B_{j+1,k} - h_{j-1,k} - B_{j-1,k}  -\frac{f}{2g} \left(v_{j+1,k} + 2 v_{j,k} + v_{j-1,k} \right) \dx \right)
	\label{eq:Kcentral}
\end{equation}
Based on Equations \eqref{eq:Kbackwards}, \eqref{eq:Kforwards} and \eqref{eq:Kcentral}, \eref{KDerivative} can be used to find $(K_x)_{j,k}$ without any recursive terms.


\subsubsection{Evaluation of $(L_y)_{j,k}$}
The numerical approximation of $(L_y)_{j,k}$ is found through \eref{pDerivatives}, as
\begin{equation}
	(L_y)_{j,k} = \mathrm{minmod} \left( \theta \frac{L_{j,k+1} - L_{j,k}}{\dy}, \frac{L_{j,k+1} - L_{j,k-1}}{2\dy}, \theta \frac{L_{j,k} - L_{j,k-1}}{\dy}\right).
	\label{eq:LDerivative}
\end{equation}
By using the per cell formula for $L$ given in \eref{KandL}, along with the recursive definition of $U_{j,k}$ given by Equations \eqref{eq:recursiveU} and \eqref{eq:Um}, the above differences can be expressed similar as for $(K_x)_{j,k}$.
The resulting forward, central and backward differences then becomes
\begin{equation}
	\theta \frac{L_{j,k+1} - L_{j,k}}{\dy} = \frac{\theta g}{\dy} \left(h_{j,k+1} + B_{j,k+1} - h_{j,k} - B_{j,k} + \frac{\dy}{2g} \left(f_{k+1}u_{j,k+1} + f_{k}u_{j,k}   \right)  \right),
	\label{eq:Lforward}
\end{equation}
\begin{equation}
	\frac{L_{j,k+1} - L_{j,k-1}}{2\dy} = \frac{g}{2\dy} \left(h_{j,k+1} + B_{j,k+1} - h_{j,k-1} - B_{j,k-1} + \frac{\dy}{2g} \left(f_{k+1}u_{j,k+1} + 2f_{k}u_{j,k} + f_{k-1}u_{j,k-1}   \right)  \right),
	\label{eq:Lcentral}
\end{equation}
and
\begin{equation}
	\theta \frac{L_{j,k} - L_{j,k-1}}{\dy} = \frac{\theta g}{\dy} \left(h_{j,k} + B_{j,k} - h_{j,k-1} - B_{j,k-1} + \frac{\dy}{2g} \left(f_{k}u_{j,k} + f_{k-1}u_{j,k-1}   \right)  \right).
	\label{eq:Lbackward}
\end{equation}


\subsubsection{Reconstruction of $h$}
In order to find the reconstruction of $h$ at cell faces without using $V$ and $U$ explicitly, consider $h_{j,k}^E$.
By inserting \eref{pOnFaces} into \eref{hOnFaces}, and using \eref{KandL}, the face value becomes
\begin{equation}
	\begin{split}
	h_{j,k}^E  &= \frac{K_{j,k}}{g} + \frac{\dx}{2g}(K_x)_{j,k} + V_{j+1/2,k} - B_{j+1/2,k} \\
			&= h_{j,k} + B_{j,k} - V_{j,k} + \frac{\dx}{2g}(K_x)_{j,k} + V_{j+1/2,k} - B_{j+1/2,k}.
	\label{eq:hE1}
	\end{split}
\end{equation}
Here, the values for $B_{j,k}$, $B_{j+1/2,k}$, $h_{j,k}$ and $(K_x)_{j,k}$ are known.
The remaining terms are then
\begin{equation}
	\begin{split}
	-V_{j,k} + V_{j+1/2,k} 	\overset{\eqref{eq:Vm}}{=}& -\frac{1}{2}\left(V_{j-1/2,k} + V_{j+1/2,k} \right) + V_{j+1/2,k} \\
						=& \frac{1}{2}\left( V_{j+1/2,k} - V_{j-1/2,k} \right) \\
						\overset{\eqref{eq:recursiveV}}{=}& \frac{1}{2}\left(  V_{j-1/2,k}  + \frac{f_k}{g}v_{j,k}\dx - V_{j-1/2,k} \right) \\
						=& \frac{f_k}{2g}v_{j,k}\dx.
	\end{split}
	\label{eq:hE2}
\end{equation}
Inserting \eref{hE2} back into \eref{hE1} results in 
\begin{equation}
	h_{j,k}^E = h_{j,k} + B_{j,k} - B_{j+1/2,k} + \frac{\dx}{2g}\left[ (K_x)_{j,k} + f_k v_{j,k} \vphantom{f^{2^2}}\right].
	\label{eq:hE}
\end{equation}
By the same derivation, the reconstructed value at the western face becomes
\begin{equation}
	h_{j,k}^W = h_{j,k} + B_{j,k} - B_{j-1/2,k} - \frac{\dx}{2g}\left[ (K_x)_{j,k} + f_k v_{j,k} \vphantom{f^{2^2}}\right].
	\label{eq:hW}
\end{equation}

The reconstruction along the $y$-axis is very similar, and starting from \eref{hOnFaces},
\begin{equation}
	\begin{split}
		h_{j,k}^N \overset{\eqref{eq:pOnFaces}}{=}& \frac{1}{g} \left( L_{j,k} + \frac{\dy}{2}(L_y)_{j,k} \right) - U_{j,k-1/2} - B_{j,k+1/2} \\
				\overset{\eqref{eq:KandL}}{=}&  h_{j,k} + B_{j,k} + U_{j,k} + \frac{\dy}{2g}(L_y)_{j,k} - U_{j,k-1/2} - B_{j,k+1/2},
	\end{split}
	\label{eq:hN1}
\end{equation}
where
\begin{equation}
	\begin{split}
		U_{j,k} - U_{j,k-1/2}  \overset{\eqref{eq:pOnFaces}}{=}& \frac{1}{2} \left(U_{j,k+1/2} + U_{j,k-1/2} \right) - U_{j,k+1/2} \\
						=& \frac{1}{2}\left(U_{j,k-1/2} - U_{j,k+1/2} \right)  \\
						\overset{\eqref{eq:recursiveU}}{=}& \frac{1}{2} \left( U_{j,k-1/2} - U_{j,k-1/2} - \frac{f_k}{g} u_{j,k} \dy \right) \\
						=& - \frac{f_k \dy}{2g} u_{j,k}.
	\end{split}
	\label{eq:hN2}
\end{equation}
This results in 
\begin{equation}
	h_{j,k}^N = h_{j,k} + B_{j,k} - B_{j,k+1/2} + \frac{\dy}{2g} \left[ (L_y)_{j,k} - f_{k} u_{j,k} \vphantom{f^{2^2}}\right].
	\label{eq:hN}
\end{equation}
Correspondingly, the southern reconstruction becomes
\begin{equation}
	h_{j,k}^S = h_{j,k} + B_{j,k} - B_{j,k-1/2} - \frac{\dy}{2g} \left[ (L_y)_{j,k} - f_{k} u_{j,k} \vphantom{f^{2^2}} \right].
	\label{eq:hN}
\end{equation}

Using the above expressions for the derivatives $(K_x)_{j,k}$ and $(L_y)_{j,k}$, and the reconstruction of $h$ at cell faces, all terms in the numerical scheme are expressed non-recursively.
The scheme is therefore suitable for implementation on massively parallel hardware, such as GPUs.


\subsection{Numerical Fluxes}

\red{Write down the central upwind fluxes as described in CDKLM16.}

\subsection{Preservation of equilibrium states}

The non-recursive CDKLM16 scheme preserves the lake-at-rest equilibrium (\red{see handwritten notes CDKLM-Coriolis 19 - 24}).
The geostrophic equilibrium given by CDKLM (1.8) is also preserved, \red{as shown in the handwritten notes CDKLM-Coriolis 24 - 29}).

	



%%%%%%%%
%%%%%%%%     Old stuff copied from the report written in Elster's PhD course
%%%%%%%%

%%%% ------ Shallow Water Equations ---- %%%%%
\section{Old text}
\subsection{The Shallow Water Equations}
\label{sec:swe}
Since we are going to apply performance models to the shallow water equations, we will here give a brief introduction to the equation and the numerical method used for solving it.
%Since this mainly serves as background information, this section is mainly taken from~\cite{MasterHolm}.
For a more thorough discussion on the topic, the books of LeVeque~\cite{LevequeFVM2004} and Toro~\cite{toroSWE2001} are recommended.

The shallow water equations, or the Saint Venant system~\cite{saintVenant} as it was originally known as, describe how waves behave under a free surface over a varying bottom topography, when motion is due to gravity only.
It is derived from the Navier-Stokes equations by assuming that the vertical component of the water velocity is negligible, compared to velocities in the horizontal plane.
The equations are given as 
\begin{equation}
	\left[ \begin{matrix} h \\ hu \\ hv \\ \end{matrix} \right]_t
	 + \left[ \begin{matrix} hu \\ hu^2 + \frac{1}{2} gh^2 \\ huv \end{matrix} \right]_x
	 + \left[ \begin{matrix} hv \\ huv \\ hv^2 + \frac{1}{2} gh^2 \end{matrix} \right]_y 
	 = \left[ \begin{matrix} 0 \\ -ghB_x \\ -ghB_y \end{matrix} \right],
	 \label{eq:app:sweasd}
\end{equation}
where $h$ is the water depth, $u$ and $v$ are water velocities in $x$ and $y$ directions respectively, and $g$ is the gravitational constant. The source term on the right hand side of the equation accounts for the bottom topography $B = [B_x, B_y]^T$. The subscripts $t$, $x$ and $y$ denote derivatives with the respective variables.

By considering a vector of conserved variables, $q = [\omega, hu, hv]^T$, where $\omega = h + B$ represent the water surface elevation, we can write Equation~\eqref{eq:app:swe} in vector form as
\begin{equation}
	q_t  + f(q)_x + g(q)_y = h_B(q).
	\label{eq:swevec}
\end{equation}
This is the conservative form of a hyperbolic partial differential equation, stating that the rate of change of the conserved variable,  $q_t$, depends on the fluxes, $f(q)_x$ and $g(q)_y$, and source terms, here caused by the bottom topography in $h_B(q)$.

Such equations are most often solved using finite volume methods~\cite{LevequeFVM2004}.
We consider a general domain $\Omega \in \mathbb{R}^2$ discretised in a regular rectangular grid with cells $\Omega_{j,k}$ defined by
\begin{equation*}
	\Omega_{j,k} =  \{ (x,y) \in \mathbb{R}^2, x_{j-1/2} \leq x \leq x_{j+1/2}, \, y_{k-1/2} \leq y \leq y_{k+1/2} \}.
	\label{eq:app:domain}
\end{equation*}
We use $\Delta x$ and $\Delta y$ as width and height for the cells respectively.
The finite volume method is derived by integrating Equation~\eqref{eq:app:vec} over the domain $\Omega_{j,k}$,
\begin{equation}
	\partder{}{t} \int_{\Omega_{j,k}} q \drm \Omega_{j,k} + \int_{\partial \Omega_{j,k}} [f \; g] \cdot \vecn \drm \gamma = \int_{\Omega_{j,k}} h_B \drm \Omega_{j,k}.
	\label{eq:app:gen}	
\end{equation}
Here, we have used the divergence theorem on the flux term, which states that the volume integral of the divergence inside a region is equal to the outward flux through the boundary of that region.
Mathematically, the divergence theorem is expressed as
\begin{equation*}
	\int_{\Omega_{j,k}} \nabla \cdot [f \; g]^T \drm \Omega_{j,k} = \int_{\partial \Omega_{j,k}} [f \; g] \cdot \vecn \drm \gamma.
	\label{eq:app:divergencetheorem}
\end{equation*}
where $\vecn$ denotes the outward unit normal vector, and $\int_{\partial \Omega_{j,k}}  \drm \gamma$ is a line integral along the boundary of $\Omega_{j,k}$.

Using the rectangular domain, we can express the line integral on the cell boundary as 
\begin{equation*}
	\begin{split}
		\int_{\partial \Omega_{j,k}} [f \; g] \cdot \vecn \drm \gamma = &\int_{y_{k-1/2}}^{y_{k+1/2}} \left[ f( q( x_{j+1/2}, y)) - f(q(x_{j-1/2}, y)) \right] \drm y \\
			&+ \int_{x_{j-1/2}}^{x_{j+1/2}} \left[ g(q(x, y_{k+1/2})) - g(q(x,y_{k-1/2})) \right] \drm x,
	\end{split}
\end{equation*}
and the integrals over $\Omega_{j,k}$ as
\begin{equation*}
	\int_{\Omega_{j,k}} q \drm \Omega_{j,k} = \int_{y_{k-1/2}}^{y_{k+1/2}} \int_{x_{j-1/2}}^{x_{j+1/2}} q \drm x \drm y.
\end{equation*}	
We can now define approximate values for $q$, $h_B$, and the flux terms for each cell, as
\begin{equation*}
	Q_{j,k} \approx \frac{1}{\Delta x \Delta y} \int_{\Omega_{j,k}} q(x,y) \drm\Omega,
\end{equation*}
\begin{equation*}
	F_{j\pm1/2, k} \approx \frac{1}{\Delta y} \int_{y_{k-1/2}}^{y_{k+1/2}} f(q(x_{j \pm 1/2}, y)) \drm y,
\end{equation*}
\begin{equation*}
	G_{j,k \pm 1/2} \approx \frac{1}{\Delta x} \int_{x_{j-1/2}}^{x_{j+1/2}} g(q( x, y_{k \pm 1/2})) \drm x,
\end{equation*}
and
\begin{equation*}
	S_{j,k} \approx \frac{1}{\Delta x \Delta y} \int_{\Omega_{j,k}} h_B(x,y) \drm \Omega.
\end{equation*}
The semi-discrete form of Equation~\eqref{eq:app:gen} then becomes
\begin{equation}
	\partder{}{t} Q_{j,k} +  \frac{F_{j+1/2, k} - F_{j-1/2,k}}{\Delta x}  + \frac{G_{j, k+1/2}  - G_{j, k-1/2}}{\Delta y} = S_{j,k}.
	\label{eq:app:disc}
\end{equation}

Equation~\eqref{eq:app:disc} is the basic form of any finite volume method.
Different schemes can be formulated by different choices of approximations of the flux terms and the source terms, as functions of $Q$.
These choices need to reflect properties of the underlying physical problem.
In the simulations considered in this report, the high-resolution Kurganov-Petrova scheme~\cite{kurganovPetrova2007} has been used.
It is well-balanced, second order accurate in the flux calculations, and accurately captures shocks in the solution without introducing artificial oscillations.
The computations in this scheme also maps well to the GPU architecture~\cite{brodtkorb_etal_10_cvs, brodtkorb_etal_11_caf}.

After finding the flux and source terms in Equation~\eqref{eq:app:disc}, it can be solved in time by a second order stability preserving Runge-Kutta method.
Let $Q_{j,k}^n$ denote the solution in cell $\Omega_{j,k}$ at time step $n$, and reorganize Equation~\eqref{eq:app:disc} to
\begin{equation*}
	\frac{\partial}{\partial t} Q_{j,k} = R(Q)_{j,k} := - \frac{F_{j+1/2, k} - F_{j-1/2, k} }{\Delta x} - \frac{G_{j, k+1/2} - G_{j, k-1/2}}{\Delta y} + S_{j,k}.
\end{equation*}
The Runge-Kutta method is then given as
\begin{equation}
	\begin{split}
		Q^*_{j,k} &= Q_{j,k}^n + \Delta t R(Q^n)_{j,k} \\
		Q^{n+1}_{j,k} &= \frac{1}{2}Q^n_{j,k} + \frac{1}{2} \left [ Q^*_{j,k} + \Delta t R(Q^*)_{j,k} \right],
	\end{split}
	\label{eq:swe:rungekutta}
\end{equation}
which solves the shallow water equations in cell $\Omega_{j,k}$ for time step $n+1$.
Here it is important to keep in mind the CFL-condition, which sets a restriction on the time step size in order to ensure a stable solution.
The time step, $\Delta t$, should be chosen so that the numerical solution can move at most one quarter grid cell per time step, meaning that
\begin{equation}
	\Delta t \leq \frac{1}{4} \min_{\Omega} \left\{ \left| \frac{\Delta x}{u \pm \sqrt{gh}} \right|, 
									     \left| \frac{\Delta y}{v \pm \sqrt{gh}} \right| \right\}.
	\label{eq:swe:CFL}
\end{equation}






%%%%%%%%%%%%%%%%%%%%%%%%%%%%%%%%%%%%%%%%%%%%%%%
\bibliography{referencesHHH}

\todos

\end{document}

